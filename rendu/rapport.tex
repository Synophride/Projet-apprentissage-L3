\documentclass{article}
\usepackage[utf8]{inputenc}
\usepackage{hyperref}

\author{Nicolas Devatine, Julien Guyot}
\title{Rendu du projet d'intro à l'IA}
\date{20 avril 2018}

\begin{document}
\maketitle
\section{Introduction}

L'archive contient : 
\begin{enumerate}
\item Les sources java, dans le dossier src. 
\item Le document présent, résumant le contenu des sources, ainsi que
  le déroulement général du projet.
\item un dossier apidocs, qui contient la javadoc du projet
\end{enumerate}


% Todo : dire de la merde

\section{Description des problèmes rencontrés au cours du projet}
\subsection{Implémentation du plateau de jeu}
Dans un premier temps, nous avions prévu un modèle assez
optimisé, qui etait refléchie pour prendre le moins de place possible,
et pour que les opérations sur le plateau soient le moins coûteuses en
CPU possibles --- Ce modèle était décrit assez précisément dans le
document du premier rendu ---


Nous avions imaginé une modélisation de l'état du plateau qui occupait
``en principe'' (c'est à dire hors questionnements spécifiques à java)
une douzaine d'octets (un long, deux shorts, et un byte). 


Néanmoins, cette idée nécessitait d'avoir un contrôle assez fin sur les
entiers, comme par exemple la gestion des entiers non signés, ou une
gestion propre des opérateurs bit à bit.


Malheureusement, la gestion de java des entiers (et/ou des shorts et bytes) de n'a
pas permis la réalisation qui était à la base prévu. Cela est
principalement dû aux spécificités de java suivantes :
\begin{itemize}
\item Les entiers java sont forcément signés --- en tous cas les
  entiers atomiques (``int'' et pas ``Integer'').
\item Les opérateurs bit à bit traitent uniquement les entiers
  (c'est à dire ni les bytes, ni les short). Par conséquent,
  l'opération (byte \& byte) (ET logique)
  \begin{enumerate}
  \item Convertit les bytes en int 
  \item Applique l'opérateur 
  \item Renvoie un int, qu'il faut re-caster si l'on cherche à calculer un byte
  \end{enumerate}
\end{itemize}
Si cela ne semble a priori pas plus problématique que cela, 
nous avons observé des effets qui étaient
insoupçonnés : Par exemple, l'opération
\((byte)  \; F0_{16} \, >>> \,4 \)  (o\`u \(F0_{16} \) est un byte), rendra
\( FF_{16} \), alors qu'on aurait pu s'attendre à \( 0F_{16} \)
\footnote{ \(X_{N}\) désigne le
  nombre X écrit en base N. L'opérateur \(>>>\) désigne le décalage à
  droite logique}


Dans un premier temps, l'optimisation semblait intéressante ---
quasiment pas d'espace gâché, des modification du plateau rapides, et
par conséquent de bonnes performances --- la réalité fut autre :


Le signage des entiers, allié au ``mono-type'' des opérateurs logiques 
causaient des effets inattendus --- De la même nature que l'exemple
évoqué ci-dessus ---. Les divers calculs qui
auraient été nécessaires pour pallier à ces problèmes auraient été
assez difficiles à implémenter, et auraient considérablement ralenti
les calculs sur le plateau. Par conséquent, ce qui aurait dû faire la
force 


Ainsi, nous avons choisi une implémentation du plateau moins
extrémiste, sans être trop différente de la première idée, afin de ne
pas avoir à réécrire trop de code. 
 
\section{Modélisation du quarto}
Nous décrirons ici l'implémentation des diverses classes composant le
package jeux.quarto
\subsection{Plateau de jeu}
\paragraph{}
La classe contenant le plateau de jeu est la classe PlateauJeu.

\paragraph{Description de la classe}
Le plateau est constitué de quatre à six attributs : Le plateau en lui-même (une
matrice de byte) qui donne l'emplacement des pièces sur le plateau, 
un attribut dénotant des pièces qui ont été jouées, un attribut (byte)
indiquant le cas échéant quelle pièce devra être jouée ainsi qu'un dernier
attribut indiquant l'état du tour (quel joueur doit jouer quel type de
coup), ainsi que les deux joueurs.

\paragraph{Modélisation des pièces}
Chaque pièce est modélisée sous la forme d'un byte, qui contient les
quatre caractéristiques définissant la pièce. Ce byte est de la forme 
\[ 0000~~couleur~hauteur~sommet~forme \]
Et tel que:
\begin{itemize}
\item couleur = 1 si la pièce est bleue, 0 si la pièce est rouge
\item hauteur = 1 si la pièce est grande, 0 si elle est petite
\item sommet = 1 si le sommet est plein, 0 sinon
\item forme = 1 si le sommet est carré, 0 si le sommet est rond
\end{itemize}

\paragraph{Attribut plateau}
Le plateau en lui-même est une matrice 4 \times 4 de bytes, o\`u
chaque case peut accueillir une pièce. Si une case donnée ne contient
pas de pièce, sa valeur est de -1.

\paragraph{Attribut piece\_a\_jouer}
L'attribut de la pièce à jouer est la pièce qu'il faut jouer, dans le
cas o\`u le tour est un dépôt de pièce. Ce dernier est initialisé à
\(-1\) , afin que lors de la première ``double-action''\footnote{ on
  appelle double-action un coup composé à la fois d'un dépôt et d'un
  don de pièce. Ainsi, une ``action'' sera uniquement soit un dépôt,
  soit un don}

\paragraph{Attribut indices\_pieces}
Cet attribut est un int, et indique quelles pièces ont été jouées, ou données.
Si le n-ème bit de poids faible est à 1, cela indique que la pièce
d'identifiant \(n-1\) a été donnée, ou posée.

\paragraph{Attribut etat\_du\_tour}
Comme son nom l'indique, cet attribut indique l'état du tour.
Son bit de poids faible est à 0 si un des joueurs doit donner une
pièce à l'autre joueur, et à 1 si l'un des joueurs doit poser une
pièce.\\
Le second bit de poids faible est à 0 si le joueur j0 doit accomplir
une action, et à 1 si c'est j1 qui doit jouer.


\paragraph{Note sur les joueurs}
Dans la classe PlateauJeu (et plus généralement dans le package
quarto), nous avons considéré que le premier joueur à jouer (j0) était
le joueur noir. \\
Par conséquent, toutes les documentations/tous les
commentaires mentionnant le joueur noir/le joueur blanc parlent
respectivement du joueur j0, et du joueur j1.

\subsection{Coups}
\subsubsection{Classe CoupQuarto}
Cette classe --- qui implémente l'interface CoupJeu ---
possède deux attributs, et est utilisée pour représenter
une action qui peut être un don comme un dépôt.

Son attribut is\_don est un booléen indiquant le type d'action
représenté : S'il est à \emph{true}, alors le coup est un don de
pièce, dans le cas contraire le coup est un dépôt.

Dans le cas o\`u l'action est un don, alors l'attribut idCoup prend la 
valeur d'un identifiant de pièce, dont la forme est décrite
ci-dessus. 

Dans le cas contraire, l'action est donc un dépôt, et idCoup
représente un identifiant d'une coordonnée, de la forme
\[ 0000~~colonne~ligne\]
O\`u la colonne est entre 0 et 3, est est codée sur 2 bits, idem pour
l'indice de la ligne.

La classe contient deux constructeurs, un à partir d'un byte et d'un
booléen, et l'autre à partir d'une chaîne de caractères, dont la forme
est spécifiée dans la documentation. % note : faire la documentation


\subsubsection{Classe DoubleCoupQuarto}
Lors des tests sur les algorithmes de jeux, nous avons remarqué que
les algorithmes de jeu partaient systématiquement du principe que les
deux joueurs jouaient l'un après l'autre.

Par conséquent, afin de ne pas avoir à modifier des algorithmes qui,
du coup, ne serait plus généraux, nous avons implémenté une classe
implémentant CoupJeu, et permettant de faire en sorte que les joueurs
jouent successivement l'un après l'autre.

Ainsi, cette classe ne contient plus un seul attribut ``identifiant'',
mais deux attributs, qui indiquent l'identifiant de la coordonnée,
et l'identifiant de la pièce.

Il y a toujours deux constructeurs, le premier par deux byte, le
second par une chaine de caractères de la forme
``[coordonnée]-[identifiant piece]''.

\subsection{Autres classes}
Contient une procédure main, qui déroule une partie

\paragraph{HeuristiqueQuarto}
Contient une heuristique pour le quarto

\paragraph{JoueurQuarto}
Implémentation de l'interface IJoueur. Contient un AlgoJeu, qui
définit quelle heuristique est adoptée.


\paragraph{JoueurHumainQuarto}
Implémente l'interface AlgoJeu, pour permettre à un humain de jouer au
quarto en tapant les coups joués

\paragraph{PartieQuarto}

\section{Heuristique}
Pour notre heuristique nous avons d'abord pris en compte les cas 
extrêmes qui sont les cas o\`u la position est finale
( ``il existe une configuration gagnante pour au moins une
caractéristique, pour l'un des deux joueurs'' ).
Dans ce cas, nous fixons une heuristique maximale ou minimale, dans
les cas o\`u la partie est respectivement gagnante ou perdante?


Il reste ensuite à traiter les autres configurations.
La difficulté du quarto est qu'en plus du plateau en lui-même,
il faut prendre en compte les pièces
non encore jouées qui vont avoir une influence sur l'heuristique.
De plus comme un coup joué est en réalité un double coup, il faut à la fois
traiter le plateau avec le dépôt qu'on a fait et le choix de la pièce
à donner.
Au quarto il est plus facile d'adopter des stratégies
spécifiques en fin de partie, notre idée est qu’un plateau avantageux
est un plateau qui va forcer l'adversaire à donner une pièce d'une
certaine caractéristique en ne lui donnant pas le choix. Pour cela il
faut mettre en ``famine'' certaines caractéristiques dans le choix des
pièces restantes tel que en plaçant une pièce possédant leur
caractéristique opposée on gagne la partie. On va alors associer une
valeur à cela pour calculer notre heuristique.
% Note : Il doit être possible de prendre en compte la ``parité'' du
% nombre de pièces restantes


La formule consiste à regarder pour chaque caractéristique s’il existe
une position gagnante sur le plateau pour cette caractéristique. Si
c'est le cas, on regarde le nombre de pièces qui n'ont pas cette
caractéristique parmi les pièces restantes. Plus ce nombre est
faible, plus on augmente la valeur de notre heuristique.
Si toutes les pièces restantes ont une caractéristique qui permet
de gagner avec le
plateau, alors l'adversaire n'aura pas le choix et il devra nous
donner la pièce qui nous fera gagner.

\section{Conclusion}

Ainsi, nous avons décrit le déroulement du projet d'introduction à
l'intelligence artificielle.

\end{document}

