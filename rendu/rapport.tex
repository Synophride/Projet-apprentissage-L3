\documentclass{article}
\usepackage[utf8]{inputenc}

\author{Nicolas Devatine, Julien Guyot}
\title{Rendu du projet d'intro à l'IA}
\date{20 avril 2018}

\begin{document}
\maketitle
\section{Introduction}

\section{Description des problèmes rencontrés au cours du projet}


% Dire que java c'est de la merde
\section{Modélisation du plateau}
Chaque pièce est modélisée sous la forme d'un byte, qui contient les
quatre caractéristiques définissant la pièce. Ce byte est de la forme 
\\ ( \( 0000~~couleur~hauteur~sommet~forme \) ) 


Le plateau de jeu est une matrice de byte, o\`u chaque case est
initialisée à -1.
L'état du tour est matérialisé dans un byte, dont seuls les deux
derniers bits sont significatifs. Celui de poids le plus faible dénote de quel
type de tour on traite, le second informe de quel joueur joue.


Le plateau stocke une pièce à jouer, qui est la pièce donnée par le
joueur.




\section{Heuristiques}

\end{document}

